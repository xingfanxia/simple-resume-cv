% !TEX TS-program = xelatex
% !TEX encoding = UTF-8 Unicode
% -*- coding: UTF-8; -*-
% vim: set fenc=utf-8

%%%%%%%%%%%%%%%%%%%%%%%%%%%%%%%%%%%%%%%%%%%%%%%%%%%%%%%%%%%%%%%%%
%% CV.tex
%% <https://github.com/zachscrivena/simple-resume-cv>
%% This is free and unencumbered software released into the
%% public domain; see <http://unlicense.org> for details.
%%%%%%%%%%%%%%%%%%%%%%%%%%%%%%%%%%%%%%%%%%%%%%%%%%%%%%%%%%%%%%%%%

% See "README.md" for instructions on compiling this document.

\documentclass[a4paper,MMMyyyy,nonstopmode]{simpleresumecv}
% Class options:
% a4paper, letterpaper, nonstopmode, draftmode
% MMMyyyy, ddMMMyyyy, MMMMyyyy, ddMMMMyyyy, yyyyMMdd, yyyyMM, yyyy

%%%%%%%%%%%%%%%%%%%%%%%%%%%%%%%%%%%%%%%%%%%%%%%%%%%%%%%%%%%%%%%%%
%% PREAMBLE.
%%%%%%%%%%%%%%%%%%%%%%%%%%%%%%%%%%%%%%%%%%%%%%%%%%%%%%%%%%%%%%%%%

% CV Info (to be customized).
\newcommand{\CVAuthor}{Xingfan Xia}
\newcommand{\CVTitle}{Xingfan's CV}
\newcommand{\CVNote}{CV compiled on {\today}}
\newcommand{\CVWebpage}{https://xiax.dev}

% PDF settings and properties.
\hypersetup{
pdftitle={\CVTitle},
pdfauthor={\CVAuthor},
pdfsubject={\CVWebpage},
pdfcreator={XeLaTeX},
pdfproducer={},
pdfkeywords={},
unicode=true,
bookmarks=true,
bookmarksopen=true,
pdfstartview=FitH,
pdfpagelayout=OneColumn,
pdfpagemode=UseOutlines,
hidelinks,
breaklinks}

% Shorthand.
\newcommand{\Code}[1]{\mbox{\textbf{#1}}}
\newcommand{\CodeCommand}[1]{\mbox{\textbf{\textbackslash{#1}}}}

%%%%%%%%%%%%%%%%%%%%%%%%%%%%%%%%%%%%%%%%%%%%%%%%%%%%%%%%%%%%%%%%%
%% ACTUAL DOCUMENT.
%%%%%%%%%%%%%%%%%%%%%%%%%%%%%%%%%%%%%%%%%%%%%%%%%%%%%%%%%%%%%%%%%
\pagestyle{empty}
\begin{document}

%%%%%%%%%%%%%%%
% TITLE BLOCK %
%%%%%%%%%%%%%%%

\Title{\CVAuthor}

\begin{SubTitle}
\href{https://www.google.com/maps/place/8502+134th+Ct+NE,+Redmond,+WA+98052}
{8502 134th Ct NE, Redmond, WA 98052}
\par
\href{mailto:xingfanxia@gmail.com}
{xingfanxia@gmail.com}
\,\SubBulletSymbol\,
+1\,(507)\,403-1689
\,\SubBulletSymbol\,
\href{\CVWebpage}
{\url{\CVWebpage}}
\end{SubTitle}

\begin{Body}
%%%%%%%%%%%%%%%%%%%%%
%% WORK EXPERIENCE %%
%%%%%%%%%%%%%%%%%%%%%

\Section
{Work\newline
Experience}
{Work Experience}
{PDF:WorkExperience}

\Entry
\href{https://aws.amazon.com/athena/}
{\textbf{Amazon Web Service -- Athena}},
Redmond, WA

\Gap
\BulletItem
Software Engineer
\hfill
\DatestampYM{2022}{4} --
Present
\begin{Detail}
\SubBulletItem
\textbf{Building the next-gen storage infrastructure for large scale distributed systems with billions of rows and terabytes of data, improving performance of AIML and other big data analytics applications.}
\SubBulletItem \textbf{Tech Stack: Java, Python, Spark, Iceberg, AWS Services(S3, CodePipeline, CodeBuild, Cloudwatch, etc.)}
\SubBulletItem
Implemented CTAS(Create Table As Select) feature for Iceberg Tables in Athena query engine. Relesed Iceberg Table GA Preview RE:INVENT 2022 and millions of CTAS queries being executed since the release.
\SubBulletItem
Overhauled the existing Hive JSON Serde library, improved the query execution performance by up to 30\% and mitigated type casting compatibility issues.
\SubBulletItem
Implemented a virtualized table connector on top of the Iceberg table connector so users can leverage the performance of the Iceberg table engine to access other table formats with limited scalability. 
Improved the query execution speed of these table formats by approximately 40\%, avoided congestion, and is projected to save 4 million in hardware costs per year.
\SubBulletItem
Worked on a cross-team endeavor to overhaul the query routing logic from Athena engine to multiple external services, reduced excessive retry behaviors and reduced service congestion by up to 32\%.
\end{Detail}

\BigGap
\Entry
\href{https://www.apple.com/}
{\textbf{Apple}},
Cupertino, CA

\Gap
\BulletItem
Machine Learning Engineer
\hfill
\DatestampYM{2020}{8} --
\DatestampYM{2022}{4}
\begin{Detail}
\SubBulletItem
\textbf{Improving users' experience with proactive AI assistant on iOS, macOS, watchOS, tvOS.}
\SubBulletItem \textbf{Tech Stack: Python, Objective-C, Sklearn Kit, Pytorch, Airflow, Hive, Presto, Spark}
\SubBulletItem
Implemented an on-device user profiling framework which consists of several models to predict users' overall interest and preferences as a upstream service which provides signal to downstream consumers like Apple Map, Apple News.
\SubBulletItem
Improved the on-device location model with 4 times of training data and better engineered aggregated features, increased user engagement rate by 9\%.
\SubBulletItem
Overhauled the topic prediction model leveraging transformer based model BERT, increased topic prediction accurarcy by 14\% and user engagement rate by 11\%.
\SubBulletItem
Developed an automatic machine learning pipeline which retrains models with data source updates, launches experiments A/B testing the models, and deploys if a specified improvement is observed, saving engineering hours across the team.
\SubBulletItem
Worked on action prediction service which predicts user intention at a particular time and location by aggregating various signals from multiple models input which is used in Siri Suggestions and Siri Shortcuts.
\end{Detail}

\BigGap
\Entry
\href{https://press.airbnb.com/about-us/}
{\textbf{Airbnb}},
San Francisco, CA

\Gap
\BulletItem
Software Engineer
\hfill
\DatestampYM{2019}{1} --
\DatestampYM{2020}{8}
\begin{Detail}
\SubBulletItem
\textbf{Fighting financial fraud with machin learning models and rule based engine.}
\SubBulletItem \textbf{Tech Stack: Python, Java, Scala, Sklearn Kit, Airflow, Hive, Presto, Spark}
\SubBulletItem
Developed a machine learning model on millions rows of data to tackle down the fake review problem at Airbnb with Sklearn and Pandas, cutting the volume of fake reviews to 1\% of the peak volume and prevented 2 million loss.
\SubBulletItem
Trained a machine learning model on millions rows of data to produce generic risk score with Sklearn and Pandas, evaluating the overall risk of users on Airbnb especially on Airbnb hosts, similar to Alipay's Zhima/Seasame score.
\SubBulletItem
Collaborated with team to launch the delayed payout policy which delays the payout of risky Airbnb hosts, prevented 60\% of fradulent payouts being made and hence in an estimated 10 mmillion in annual savings.
\SubBulletItem
Built a java microservice which provides downstream services with consolidated risk signals using Redis, Hive, and Kafka.
\SubBulletItem
Designed and Implemented a scoring java service which produces risk score of certain event and senario supporting 6 downstream consumers using Redis, Hive, and Kafka.
\end{Detail}

\BigGap
\Entry
\href{https://lino.network/}
{\textbf{Lino Network}},
Cupertino, CA

\Gap
\BulletItem
Founding Engineer
\hfill
\DatestampYM{2018}{03} --
\DatestampYM{2018}{10}

% \newline
% Building a decentrailized video and live streaming platform.
\begin{Detail}
\SubBulletItem \textbf{Building a decentrailized video and live streaming platform -- Dlive.}
\SubBulletItem \textbf{Tech Stack: Python, Golang, Javascript, VueJS, Splunk, ELK, AWS(S3, Kinesis, Lambda, Route53, etc)}
\SubBulletItem 
Built Dlive's front end website, video transcoding/storage/distribution service, and metrics data pipeline from scratch.
\SubBulletItem
Secured \$20 million in funding through a private token sale led by ZhenFund and exited selling Dlive to Bittorrent.
\end{Detail}

% \BigGap
% \Entry
% \href{https://hq.vevo.com/}
% {\textbf{Vevo LLC}},
% San Francisco, CA

% \Gap
% \BulletItem
% Software Engineer Intern
% \hfill
% \DatestampYM{2017}{06} --
% \DatestampYM{2017}{09}
% \newline
% (Data \& Machine Learning Team)
% \begin{Detail}
% \SubBulletItem
% Designed and implemented a Q\&A chatbot using AWS Lambda, Python and AWS Lex. Deployed the chatbot on Facebook Messenger and Slack.
% \SubBulletItem
% Built a multi-seed recommendation engine. Deployed as an API service, a Web App, and a tvOS App as POC demos.
% \SubBulletItem
% Analyzed visuals of music videos, generated playlists based on similar visual elements grouped by LDA topic modeling.
% \SubBulletItem
% Deployed a Jupyterhub system on an AWS P2 instance to facilitate team's research and exploratory works related to machine learning and deep learning. Typically made model training 6 times faster.
% \end{Detail}

%%%%%%%%%%%%%%
%% Projects %%
%%%%%%%%%%%%%%

% \Section
% {Projects}
% {Projects}
% {PDF:Projects}

% \Entry
% \href{https://github.com/xingfanxia/comps_ediscovery}
% {\textbf{Tree Discovery}}
% % \newline
% % (An implementation of the Random Forest Algorithm)
% \Gap
% \BulletItem
% Lead
% \hfill
% \DatestampYM{2017}{10} --
% \DatestampYM{2018}{05}

% \begin{Detail}
% \SubBulletItem
% Implemented the Random Forest Algorithm from scratch in Python with a team of 6.
% \SubBulletItem
% Processed the raw and messy ENRON email dataset into a usable training set and test set with TF-IDF, DB-SCAN, and LSA topic aggregation.
% % \SubBulletItem
% % Reduced the training speed of a typical tree from 2 minutes down to less than 10 seconds.
% \SubBulletItem
% Tested the performance of the implementation against the ENRON dataset, our F1 Score is comparable to the results presented at TREC Conference 2011.
% \SubBulletItem
% Designed and built a demo e-discovery product that utilizes the idea of semi-supervised learning, allow users to tag relevant documents as training set in each iteration of predictions to continuously improve the model's accuracy and recall.
% \end{Detail}

% \BigGap
% \Entry
% \href{https://github.com/mahu-search}
% {\textbf{Mahu}}

% \Gap
% \BulletItem
% Contributor
% \hfill
% \DatestampYM{2017}{07} --
% \DatestampYM{2018}{03}

% \begin{Detail}
% \SubBulletItem
% Built a web app with VueJS and bulma that allows developers from China to perform searches related to coding and the app will crawl aggregate results from sites like StackOverflow, SuperUser, Stack Exchange, etc. 
% \SubBulletItem
% Integrated various translation API avialble to empower users who are not proficient with English.
% \end{Detail}

% \BigGap

% \Entry
% \href{https://moegirl.org/Mainpage}
% {\textbf{Moegirl Wiki}}
% \newline
% (Open Source Wiki Based on MediaWiki)
% \Gap
% \BulletItem
% Contributor
% \hfill
% \DatestampYM{2016}{03} --
% \DatestampYM{2018}{03}
% \begin{Detail}
% \SubBulletItem
% Worked with a team of about 20 contributors to maintain day-to-day running of the Wiki.
% \SubBulletItem
% Prototyped a redesign of the main page of the iOS app with the lead iOS developer.
% \end{Detail}

% \BigGap
% \Entry
% \href{https://steamcommunity.com/sharedfiles/filedetails/?id=440115357}
% {\textbf{Dota IMBA}}
% \newline
% (Custom Game Based on Popular Online Multiplayer Game Dota2 with Six Million Subscribers)

% \Gap
% \BulletItem
% Contributor
% \hfill
% \DatestampYM{2017}{07} --
% \DatestampYM{2018}{03}
% \begin{Detail}
% \SubBulletItem
% Applied patches for community reported bug and adjust parameters for better balance.
% \SubBulletItem
% Communicated with active players on discord channels to collect critiques and ideas for improvement.
% \end{Detail}


% \BigGap
% \Entry
% \href{https://github.com/xxf1995/VRGame}
% {\textbf{Zombie Forest}}

% \Gap
% \BulletItem
% Contributor
% \hfill
% \DatestampYM{2016}{6} --
% \DatestampYM{2016}{6} --
% \begin{Detail}
% \SubBulletItem
% Learned basics of Unity and Oculus SDK and built the game from scratch in 36 hours at CarlHack 2016.
% \SubBulletItem
% Designed the basic gameplay and AI logic, implemented how player character interact with the map.
% \end{Detail}

%%%%%%%%%%%%%%%
%% EDUCATION %%
%%%%%%%%%%%%%%%

\Section
{Education}
{Education}
{PDF:Education}

\Entry
\href{https://www.carleton.edu/}
{\textbf{Carleton College}},
Northfield, Minnesota

\Gap
\BulletItem
B.A. in
\href{https://apps.carleton.edu/curricular/cs/}
{Computer Science}
\hfill
Graduated \DatestampYM{2018}{06}

%%%%%%%%%%%%%%%%%%%%%
%%Technical SKILLS %%
%%%%%%%%%%%%%%%%%%%%%

\Section
{Technical Skills}
{Technical Skills}
{PDF:Technical Skills}

\BulletItem
Programming Languages: Proficiency in Python, Java, Golang, Javascript, Scala, Objective-C

\Gap
\BulletItem
Experienced AWS Services: Athena, S3, SQS, EC2, Lambda, API Gateway, Auto Scaling Group, Cloudfront, Route 53, CodePipeline, CodeBuild, CodeDeploy, Kinesis

\Gap
\BulletItem
Machine Learning \& Data Engineering: Scikit Learn, OpenCV, Pytorch, Matplotlib, Scipy, Numpy, Pandas, ELK, Apache Iceberg, Apache Spark, Apache Kafka, Airflow

\end{Body}
\end{document}
