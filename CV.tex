% !TEX TS-program = xelatex
% !TEX encoding = UTF-8 Unicode
% -*- coding: UTF-8; -*-
% vim: set fenc=utf-8

%%%%%%%%%%%%%%%%%%%%%%%%%%%%%%%%%%%%%%%%%%%%%%%%%%%%%%%%%%%%%%%%%
%% CV.tex
%% <https://github.com/zachscrivena/simple-resume-cv>
%% This is free and unencumbered software released into the
%% public domain; see <http://unlicense.org> for details.
%%%%%%%%%%%%%%%%%%%%%%%%%%%%%%%%%%%%%%%%%%%%%%%%%%%%%%%%%%%%%%%%%

% See "README.md" for instructions on compiling this document.

\documentclass[letterpaper,MMMyyyy,nonstopmode]{simpleresumecv}
% Class options:
% a4paper, letterpaper, nonstopmode, draftmode
% MMMyyyy, ddMMMyyyy, MMMMyyyy, ddMMMMyyyy, yyyyMMdd, yyyyMM, yyyy

%%%%%%%%%%%%%%%%%%%%%%%%%%%%%%%%%%%%%%%%%%%%%%%%%%%%%%%%%%%%%%%%%
%% PREAMBLE.
%%%%%%%%%%%%%%%%%%%%%%%%%%%%%%%%%%%%%%%%%%%%%%%%%%%%%%%%%%%%%%%%%

% CV Info (to be customized).
\newcommand{\CVAuthor}{Xingfan Xia}
\newcommand{\CVTitle}{Xingfan's CV}
\newcommand{\CVNote}{CV compiled on {\today}}
\newcommand{\CVWebpage}{https://xiax.dev}

% PDF settings and properties.
\hypersetup{
pdftitle={\CVTitle},
pdfauthor={\CVAuthor},
pdfsubject={\CVWebpage},
pdfcreator={XeLaTeX},
pdfproducer={},
pdfkeywords={},
unicode=true,
bookmarks=true,
bookmarksopen=true,
pdfstartview=FitH,
pdfpagelayout=OneColumn,
pdfpagemode=UseOutlines,
hidelinks,
breaklinks}

% Shorthand.
\newcommand{\Code}[1]{\mbox{\textbf{#1}}}
\newcommand{\CodeCommand}[1]{\mbox{\textbf{\textbackslash{#1}}}}

%%%%%%%%%%%%%%%%%%%%%%%%%%%%%%%%%%%%%%%%%%%%%%%%%%%%%%%%%%%%%%%%%
%% ACTUAL DOCUMENT.
%%%%%%%%%%%%%%%%%%%%%%%%%%%%%%%%%%%%%%%%%%%%%%%%%%%%%%%%%%%%%%%%%
\pagestyle{empty}
\begin{document}

%%%%%%%%%%%%%%%
% TITLE BLOCK %
%%%%%%%%%%%%%%%

\Title{\CVAuthor}

\begin{SubTitle}
\href{https://www.google.com/maps/place/855+Brannan,+San+Francisco,+CA+94103}
{855 Brannan Street, San Francisco CA 94103}
\par
\href{mailto:xingfanxia@gmail.com}
{xingfanxia@gmail.com}
\,\SubBulletSymbol\,
+1\,(507)\,403-1689
\,\SubBulletSymbol\,
\href{\CVWebpage}
{\url{\CVWebpage}}
\end{SubTitle}

\begin{Body}

%%%%%%%%%%%%%%%
%% EDUCATION %%
%%%%%%%%%%%%%%%

\Section
{Education}
{Education}
{PDF:Education}

\Entry
\href{https://www.carleton.edu/}
{\textbf{Carleton College}},
Northfield, Minnesota

\Gap
\BulletItem
B.A. in
\href{https://apps.carleton.edu/curricular/cs/}
{Computer Science}
\hfill
\DatestampYM{2017}{03}
\begin{Detail}
\SubBulletItem
Related Course Works: Human-centered Computing, Operating Systems, Database Systems, Computer Graphics, Software Design, Programming Languages, Computability and Complexity, Coding Theory, Computer Sound and Music

\end{Detail}

%%%%%%%%%%%%%%%%%%%%%
%% WORK EXPERIENCE %%
%%%%%%%%%%%%%%%%%%%%%

\Section
{Work\newline
Experience}
{Work Experience}
{PDF:WorkExperience}
\Entry
\href{https://press.airbnb.com/about-us/}
{\textbf{Airbnb}},
San Francisco, CA

\Gap
\BulletItem
Software Engineer
\hfill
\DatestampYM{2019}{1} --
Present
\begin{Detail}
\SubBulletItem
Developed a machine learning model on millions rows of data to tackle down the fake review problem at Airbnb with Sklearn and Pandas, cutting the volume of fake reviews to 1\% of the peak volume and prevented 2 million loss in coupon and credits.
\SubBulletItem
Trained a machine learning model on millions rows of data to produce generic risk score with Sklearn and Pandas, evaluating the overall risk of users on Airbnb especially on Airbnb hosts, similar to Alipay's Zhima/Seasame score.
\SubBulletItem
Collaborated with team to launch the delayed payout policy which delays the payout of risky Airbnb hosts, prevented 60\% of fradulent payouts being made and hence in an estimated 10 mmillion in annual savings.
\SubBulletItem
Built a java service which provides downstream service with risk signals easily from a single source of truth wihout having to dive into hundreds of services and APIs using Redis, Hive, and Apache Kafka.
\SubBulletItem
Designed and Implemented a scoring java service which produces risk score of certain events and senario supporting 6 downstream consumers using Redis, Hive, and Apache Kafka.
\end{Detail}

\BigGap
\Entry
\href{https://lino.network/}
{\textbf{Lino Network}},
Cupertino, CA

\Gap
\BulletItem
Software Engineer
\hfill
\DatestampYM{2018}{03} --
\DatestampYM{2018}{10}

\newline
(Platform \& Infrastructure)
\begin{Detail}
\SubBulletItem
Performed a major upgrade of Dlive, largest live streaming DApp on Blockchain; implemented a discord-like feature with VueJS, GraphQL, and Web Socket. Dramatically increased user retention rate.
\SubBulletItem
Architected and developed a video platform that includes ingest, transcoding, and distribution of video files in Golang with AWS S3, SQS, and Cloudfront from scratch. Serving terabytes of video content in different qualities globally.
\SubBulletItem
Built GPU-accelerated transcoding cluster for the video platform in Golang with FFMPEG and Nvidia CUDA. Increased transcoding efficiency by 15 times and expanded the supported video formats and codecs.
\SubBulletItem
Improved users' experience uploading video by deploying video ingest service in 7 major regions of the world to reduce uploading wait time and the possibility of transfer errors with AWS API Gateway and AWS Lambda.
\SubBulletItem
Secured content on the Dlive platform by implementing signed-cookie protection on AWS Clooudfront in Python to defend third-party malicious attack.
\SubBulletItem
Designed and implemented data pipeline at Lino from scratch to monitor user behaviors, user QoE metrics, and streaming service robustness with AWS Kinesis, AWS Lambda, ELK Stack, and Splunk. Performed real-time analytics of gigabytes of logs from users and aggregated into comprehensive reports and dashboards that provides both technical and business insides.
\SubBulletItem
Researched popular live streaming protocols including HLS, MPEG-DASH, Web-RTC and prototyped their implementations.
\end{Detail}

\BigGap
\Entry
\href{https://hq.vevo.com/}
{\textbf{Vevo LLC}},
San Francisco, CA

\Gap
\BulletItem
Software Engineer Intern
\hfill
\DatestampYM{2017}{06} --
\DatestampYM{2017}{09}
\newline
(Data \& Machine Learning Team)
\begin{Detail}
\SubBulletItem
Designed and implemented a Q\&A chatbot using AWS Lambda, Python and AWS Lex. Deployed the chatbot on Facebook Messenger and Slack.
\SubBulletItem
Built a multi-seed recommendation engine. Deployed as an API service, a Web App, and a tvOS App as POC demos.
\SubBulletItem
Analyzed visuals of music videos, generated playlists based on similar visual elements grouped by LDA topic modeling.
\SubBulletItem
Deployed a Jupyterhub system on an AWS P2 instance to facilitate team's research and exploratory works related to machine learning and deep learning. Typically made model training 6 times faster.
\end{Detail}

%%%%%%%%%%%%%%
%% Projects %%
%%%%%%%%%%%%%%

% \Section
% {Projects}
% {Projects}
% {PDF:Projects}

% \Entry
% \href{https://github.com/xingfanxia/comps_ediscovery}
% {\textbf{Tree Discovery}}
% % \newline
% % (An implementation of the Random Forest Algorithm)
% \Gap
% \BulletItem
% Lead
% \hfill
% \DatestampYM{2017}{10} --
% \DatestampYM{2018}{05}

% \begin{Detail}
% \SubBulletItem
% Implemented the Random Forest Algorithm from scratch in Python with a team of 6.
% \SubBulletItem
% Processed the raw and messy ENRON email dataset into a usable training set and test set with TF-IDF, DB-SCAN, and LSA topic aggregation.
% % \SubBulletItem
% % Reduced the training speed of a typical tree from 2 minutes down to less than 10 seconds.
% \SubBulletItem
% Tested the performance of the implementation against the ENRON dataset, our F1 Score is comparable to the results presented at TREC Conference 2011.
% \SubBulletItem
% Designed and built a demo e-discovery product that utilizes the idea of semi-supervised learning, allow users to tag relevant documents as training set in each iteration of predictions to continuously improve the model's accuracy and recall.
% \end{Detail}

% \BigGap
% \Entry
% \href{https://github.com/mahu-search}
% {\textbf{Mahu}}

% \Gap
% \BulletItem
% Contributor
% \hfill
% \DatestampYM{2017}{07} --
% \DatestampYM{2018}{03}

% \begin{Detail}
% \SubBulletItem
% Built a web app with VueJS and bulma that allows developers from China to perform searches related to coding and the app will crawl aggregate results from sites like StackOverflow, SuperUser, Stack Exchange, etc. 
% \SubBulletItem
% Integrated various translation API avialble to empower users who are not proficient with English.
% \end{Detail}

% \BigGap

% \Entry
% \href{https://moegirl.org/Mainpage}
% {\textbf{Moegirl Wiki}}
% \newline
% (Open Source Wiki Based on MediaWiki)
% \Gap
% \BulletItem
% Contributor
% \hfill
% \DatestampYM{2016}{03} --
% \DatestampYM{2018}{03}
% \begin{Detail}
% \SubBulletItem
% Worked with a team of about 20 contributors to maintain day-to-day running of the Wiki.
% \SubBulletItem
% Prototyped a redesign of the main page of the iOS app with the lead iOS developer.
% \end{Detail}

% \BigGap
% \Entry
% \href{https://steamcommunity.com/sharedfiles/filedetails/?id=440115357}
% {\textbf{Dota IMBA}}
% \newline
% (Custom Game Based on Popular Online Multiplayer Game Dota2 with Six Million Subscribers)

% \Gap
% \BulletItem
% Contributor
% \hfill
% \DatestampYM{2017}{07} --
% \DatestampYM{2018}{03}
% \begin{Detail}
% \SubBulletItem
% Applied patches for community reported bug and adjust parameters for better balance.
% \SubBulletItem
% Communicated with active players on discord channels to collect critiques and ideas for improvement.
% \end{Detail}


% \BigGap
% \Entry
% \href{https://github.com/xxf1995/VRGame}
% {\textbf{Zombie Forest}}

% \Gap
% \BulletItem
% Contributor
% \hfill
% \DatestampYM{2016}{6} --
% \DatestampYM{2016}{6} --
% \begin{Detail}
% \SubBulletItem
% Learned basics of Unity and Oculus SDK and built the game from scratch in 36 hours at CarlHack 2016.
% \SubBulletItem
% Designed the basic gameplay and AI logic, implemented how player character interact with the map.
% \end{Detail}


%%%%%%%%%%%%%%%%%%%%%
%%Technical SKILLS %%
%%%%%%%%%%%%%%%%%%%%%

\Section
{Technical Skills}
{Technical Skills}
{PDF:Technical Skills}

\BulletItem
Programming Languages: Proficiency in Golang, Python, Java, Javascript, Scala, C

\Gap
\BulletItem
Frameworks: VueJS, Bulma, LAMP, GraphQL, Selenium, Webpack, NodeJS, PhantomJS, React, Django, ELK, Splunk, Apache Spark, Apache Kafka

\Gap
\BulletItem
Tools \& Tools: Ubuntu, Docker, FFMPEG, Nvidia CUDA, MongoDB, MySQL, Redis, Hive, HDFS, Presto

\Gap
\BulletItem
Proficient AWS Services: S3, SQS, EC2, Lambda, API Gateway, Auto Scaling Group, Cloudfront, Route 53, CodePipeline, CodeBuild, CodeDeploy, Elastic Beanstalk

\Gap
\BulletItem
Machine Learning \& Deep Learning: Scikit Learn, OpenGL, Theano, Matplotlib, Scipy, Numpy, Pandas, Keras, Jupyter

\end{Body}
\end{document}
